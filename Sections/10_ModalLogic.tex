\section{Modal logic}

\subsection{Syntax}

In modal logic, formulas are constructed by applying negation, conjunction, disjunction, implication, as well as the box and diamond operators to propositions.
%
\begin{align*}
    \text{proposition} &:= p \;|\; q \;|\; r \cdots\\
    \text{formula} &:= \text{proposition} \;|\; \neg\text{formula} \;|\; (\text{formula} \circ \text{formula}) \;|\; \Box \text{formula} \;|\; \Diamond \text{formula} \tag{where \(\circ\) is \(\land\), \(\lor\) or \(\rightarrow\)}
\end{align*}


\subsection{Semantics}

A \emph{Kripke frame} \(\mathcal{F} = (W, R)\) contains a set \(W\) of worlds and a binary relation \(R \subseteq W \times W\). This can be represented as a directed graph with nodes \(W\) and edges \(R\).


\begin{figure}[H]
    \centering
    \begin{tikzpicture}[scale=1.275]
        \node (w1) at (0, 0)[draw=MidnightBlue, circle, very thick, MidnightBlue, fill=MidnightBlue!20] {\(w_1\)};

        \node (w2) at (2, 0)[draw=MidnightBlue, circle, very thick, MidnightBlue, fill=MidnightBlue!20] {\(w_2\)};

        \node (w3) at (1, -1.5)[draw=MidnightBlue, circle, very thick, MidnightBlue, fill=MidnightBlue!20] {\(w_3\)};

        \node (w4) at (3, -1.5)[draw=MidnightBlue, circle, very thick, MidnightBlue, fill=MidnightBlue!20] {\(w_4\)};

        \node (w5) at (4, 0)[draw=MidnightBlue, circle, very thick, MidnightBlue, fill=MidnightBlue!20] {\(w_5\)};

        \draw[black, thick, -Latex] (w1) -- (w2);
        \draw[black, thick, -Latex] (w2) -- (w3);
        \draw[black, thick, -Latex] (w2) -- (w5);
        \draw[black, thick, -Latex] (w3) -- (w1);
        \draw[black, thick, -Latex] (w4) -- (w5);
        \draw[black, thick, -Latex] (w5.45) arc (-45:225:10pt);
    \end{tikzpicture}
    \caption{A Kripke frame \(\mathcal{F} = (W, R)\) with worlds \(W = \{w_1, w_2, w_3, w_4, w_5\}\) and the relation \(R = \{(w_1, w_2), (w_2, w_3), (w_2, w_5), (w_3, w_1), (w_4, w_5)\}\).}
    \label{fig:Ch10-frame-example}
\end{figure}

A valuation \(V\) is a function that maps each propositional letter to a subset of \(W\). For example, we may have
%
\begin{align*}
    V(p) &= \{w_1, w_3, w_5\}\\
    V(q) &= \{w_1, w_2\}\\
    V(r) &= \emptyset
\end{align*}
%
A Kripke frame \((W, R)\) combined with a valuation \(V\) gives a \emph{Kripke model} \(\mathcal{M} = (W, R, V)\).

Modal logic is a local logic where formulas are evaluated not just with a model, but at a specific world as well. For any given world \(w \in W\), we define the semantics of modal logic as
%
\begin{align*}
    \mathcal{M}, w \models p &\iff w \in V(p)\\
    \mathcal{M}, w \models \neg\phi &\iff \mathcal{M}, w \not\models \phi\\
    \mathcal{M}, w \models (\phi\land\psi) &\iff \mathcal{M}, w \models \phi \text{ and } \mathcal{M}, w \models \psi\\
    \mathcal{M}, w \models \Diamond\phi &\iff \text{there exists some } w' \in W \text{ such that } (w, w') \in R \text{ and } \mathcal{M}, w' \models\phi\\
    \mathcal{M}, w \models \Box\phi &\iff \text{for all } w' \in W, \text{ if } (w, w') \in R \text{ then } \mathcal{M}, w' \models\phi
\end{align*}
%
where \(p\) is a propositional letter and \(\phi\) and \(\psi\) are formulas.

A formula may be valid in a model, in a frame, or over a class of frames.
%
\begin{align*}
    (W, R, V) \models \phi &\iff (W, R, V), w \models \phi \text{ for all } w \in W \tag{validity in a model}\\
    (W, R) \models \phi &\iff (W, R, V) \models \phi \text{ for all valuations } V \tag{validity in a frame}\\
    \mathcal{K} \models \phi &\iff \mathcal{F} \models \phi \text{ for all frames } \mathcal{F}\in\mathcal{K} \tag{validity over a class of frames}
\end{align*}
%
Any formula that is valid in propositional logic is also valid in modal logic. Other valid formulas include
%
\begin{align*}
    \Box(p \land q) &\rightarrow (\Box p \land \Box q)\\
    \Box(p \rightarrow q) &\rightarrow (\Box p \rightarrow \Box q)
\end{align*}
%
but not
%
\[\Box(p \lor q) \rightarrow (\Box p \lor \Box q)\text{.}\]


\subsection{Axiomatic proof system}

In addition to the axioms for propositional logic, an axiomatic proof system for modal logic requires the following axiom.
%
\[\Box(p \rightarrow q) \rightarrow (\Box p \rightarrow \Box q)\]
%
We also use the following inference rules.
%
\[
    \infer{\psi}{
        \phi
        &
        (\phi\rightarrow\psi)
    }
    %
    \tag{modus ponens}
\]
%
\[\infer{\Box\phi}{\phi} \tag{necessitation}\]



\subsection{Classes of frames, soundness and completeness}

Frames with common properties may be grouped into a class \(\mathcal{K}\). We say that a formula \(\phi\) \emph{defines} \(\mathcal{K}\) if
%
\[\mathcal{F} \vdash \phi \iff \mathcal{F}\in\mathcal{K}\text{.}\]
%
Table \ref{tab:Ch10-defining-classes} shows several examples of such classes and their defining modal formulas. Moreover, names are often assigned to classes with special properties, as shown in Table \ref{tab:Ch10-class-names}.

\begin{table}[H]
    \centering
    \begin{tabular}{|c|c|c|}
        \hline
        \textbf{Class of...} & \textbf{First-order definition} & \textbf{Defining modal formula}\\
        \hline
        Reflexive frames & \(\forall w\; Rww\) & \(\Box p \rightarrow p\)\\
        \hline
        Transitive frames & \(\forall u\; \forall v\; \forall w\; ((Ruv \land Rvw) \rightarrow Ruw)\) & \(\Diamond\Diamond p \rightarrow \Diamond p\) or \(\Box p \rightarrow \Box\Box p\)\\
        \hline
        Symmetric frames & \(\forall u\; \forall v\; (Ruv \rightarrow Rvu)\) & \(p \rightarrow \Box\Diamond p\)\\
        \hline
        Dense frames & \(\forall u\; \forall v\; (Ruv \rightarrow \exists w\; (Ruw \land Rwv))\) & \(\Diamond p \rightarrow \Diamond\Diamond p\) or \(\Box\Box p \rightarrow \Box p\)\\
        \hline
    \end{tabular}
    \caption{Classes of Kripke frames and the modal formulas that define them.}
    \label{tab:Ch10-defining-classes}
\end{table}

\begin{table}[H]
    \centering
    \begin{tabular}{|c|c|}
        \hline
        \textbf{Name} & \textbf{Class of...}\\
        \hline
        \(K\) & All frames\\
        \hline
        \(T\) & Reflexive frames\\
        \hline
        \(S4\) & Reflexive and transitive frames\\
        \hline
        \(S5\) & Frames with equivalence relations (reflexive, symmetric and transitive)\\
        \hline
    \end{tabular}
    \caption{Classes of Kripke frames and their names.}
    \label{tab:Ch10-class-names}
\end{table}

For a class \(\mathcal{K}\) of frames, let \(A\) be the conjunction of its axioms and defining formulas. For any formula \(\phi\), we write \(\vdash_A \phi\) if \(\phi\) is provable using \(A\) through modus ponens and necessitation. It follows that
%
\[\mathcal{K} \models \phi \iff \vdash_A \phi\]
%
meaning that \(\vdash_A\) is sound and complete for \(\mathcal{K}\).



\subsection{Modal tableaus}

Like in propositonal and first-order logic, the satisfiability modal formulas can be verified with tableaus. 

The tableau will consist of a queue of \emph{labelled frames}. A labelled frame \(((W, R), \lambda)\) contains a function \(\lambda\) which maps each world \(W\) to a set of modal formulas. We may visualise this as a frame where each world is labelled with zero or more formulas.

The following algorithm is used to determine the satisfiability of a formula \(\phi\).
%
{
\small
\begin{lstlisting}[language=python, commentstyle=\color{gray}]
def is_satisfiable($\phi$):
    Tableau = Queue()
    Tableau.enqueue(frame containing only one world labelled $\phi$)

    while Tableau is not empty:
        # Dequeue a labelled frame from the tableau
        $((W, R), \lambda)$ = Tableau.dequeue()

        if $\{p, \neg p\} \subseteq \lambda(w)$ for some $w \in W$ and propositional letter $p$:
            # There is a world with contradictory literals,
            # so don't enqueue this frame back
            continue 
        
        if for all $w \in W$, each formula $\theta \in \lambda(w)$ is a literal, box or negated diamond:
            return True

        select a formula $\theta\in\lambda(w)$ ($w\in W$) that is not a literal, box or negated diamond

        if $\theta$ is an $\alpha$-formula:
            let $\lambda'$ be identical to $\lambda$ except $\lambda'(w) = (\lambda(w)\setminus\{\theta\})\cup\{\alpha_1, \alpha_2\}$
            Tableau.enqueue($((W, R), \lambda')$)
        
        elif $\theta$ is a $\beta$-formula:
            let $\lambda_1$ be identical to $\lambda$ except $\lambda_1(w) = (\lambda(w)\setminus\{\theta\})\cup\{\beta_1\}$
            Tableau.enqueue($((W, R), \lambda_1)$)

            let $\lambda_2$ be identical to $\lambda$ except $\lambda_2(w) = (\lambda(w)\setminus\{\theta\})\cup\{\beta_2\}$
            Tableau.enqueue($((W, R), \lambda_2)$)
        
        elif $\theta = \Diamond A$:
            let $W' = W \cup \{w_{\text{new}}\}$ with a new world $w_{\text{new}}$
            let $R' = R \cup \{(w, w_\text{new})\}$
            let $\lambda'$ be identical to $\lambda$ except $\lambda'(w_\text{new}) = \{A\}\cup\{B : \Box B \in\lambda(w)\}\cup\{\neg B: \neg\Diamond B \in \lambda(w)\}$
                                           and $\lambda'(w) = \lambda(w)\setminus\{\theta\}$
            Tableau.enqueue($((W', R'), \lambda')$)
        
        elif $\theta = \neg\Box A$:
            let $W' = W \cup \{w_{\text{new}}\}$ with a new world $w_{\text{new}}$
            let $R' = R \cup \{(w, w_\text{new})\}$
            let $\lambda'$ be identical to $\lambda$ except $\lambda'(w_\text{new}) = \{\neg A\}\cup\{B : \Box B \in\lambda(w)\}\cup\{\neg B: \neg\Diamond B \in \lambda(w)\}$
                                           and $\lambda'(w) = \lambda(w)\setminus\{\theta\}$
            Tableau.enqueue($((W', R'), \lambda')$)
    
    return False
\end{lstlisting}
}

We may adapt this tableau algorithm for determining satisfiability in specific classes of frames.
%
\begin{itemize}
    \item To determine satisfiability of a formula \(\phi\) in reflexive frames, initialise the tableau with a frame \((W, R, \lambda)\) where \(W = \{w\}\), \(R = \{(w, w)\}\) and \(\lambda(w) = \phi\). Construct the tableau as usual. Whenever a new world \(w_\text{new}\) is added:
    \begin{itemize}
        \item add \((w_\text{new}, w_\text{new})\) to \(R\);
        \item if \(\Box A \in \lambda(w_\text{new})\), also include \(A \in \lambda(w_\text{new})\); and
        \item if \(\neg\Diamond A \in \lambda(w_\text{new})\), also include \(\neg A \in \lambda(w_\text{new})\).
    \end{itemize}

    \item To determine satisfiability of a formula in symmetric frames, construct the tableau as usual. For diamond formulas in world \(w\), when a new world \(w_\text{new}\) is added with a new edge \((w, w_\text{new})\), also include the edge \((w_\text{new}, w)\). Any boxed or negated diamond formulas in \(w_\text{new}\) should propagate back to \(w\).
    
    \item To determine satisfiability of a formula in transitive frames, construct the tableau as usual. For diamond formulas in world \(w\), when a new world \(w_\text{new}\) is added with a new edge \((w, w_\text{new})\), then
    \begin{itemize}
        \item add the edge \((v, w_\text{new})\);
        \item if \(\Box A \in \lambda(v)\), include \(A \in \lambda(w)\); and
        \item if \(\neg\Diamond A \in \lambda(v)\), include \(\neg A \in \lambda(w)\)
    \end{itemize}
    for each world \(v\) that has an outgoing edge to \(w\). Note that tableaus for transitive models may not terminate.
    
    
\end{itemize}



\subsection{Frame and model p-morphisms}

Let \((W, R)\) and \((W', R')\) be frames. A function \(f : W \rightarrow W'\) is called a \emph{frame p-morphism} if
%
\begin{itemize}
    \item \((x, y) \in R\) implies \((f(x), f(y)) \in R'\) (a homomorphism); and
    \item if \((f(x), y') \in R'\), then there is some \(y \in W\) such that \(f(y) = y'\) and \((x, y) \in R\).
\end{itemize}
%
Below shows an example of a frame p-morphism between two frames \((W, R)\) and \((W', R')\), with a function \(f\) mapping worlds from \(W = \{A, B, C, D, E\}\) to worlds from \(W' = \{X, Y, Z\}\). Notice that
%
\begin{itemize}
    \item Edges are retained by the mapping. If there are two worlds in \(W\) connected by an edge, they must remain connected after the mapping.
    \item If an edge connects two worlds in \(W'\), one of which can be ``traced back'' through \(f\) to a world \(w \in W\), then it must also be possible to trace back the other world to some \(w' \in W\) connected to \(w\).
\end{itemize}

\begin{figure}[H]
    \centering
    \begin{tikzpicture}
        \node (A) at (0, 0)[draw=MidnightBlue, circle, very thick, MidnightBlue, fill=MidnightBlue!20] {\(A\)};

        \node (B) at (2, 0)[draw=MidnightBlue, circle, very thick, MidnightBlue, fill=MidnightBlue!20] {\(B\)};

        \node (C) at (0, -2)[draw=MidnightBlue, circle, very thick, MidnightBlue, fill=MidnightBlue!20] {\(C\)};

        \node (D) at (2, -2)[draw=MidnightBlue, circle, very thick, MidnightBlue, fill=MidnightBlue!20] {\(D\)};

        \node (E) at (1, -1)[draw=MidnightBlue, circle, very thick, MidnightBlue, fill=MidnightBlue!20] {\(E\)};


        \draw[black, thick, Latex-Latex] (A) -- (C);
        \draw[black, thick, -Latex] (A) -- (E);
        \draw[black, thick, Latex-Latex] (B) -- (E);
        \draw[black, thick, Latex-Latex] (B) -- (D);
        \draw[black, thick, -Latex] (C) -- (E);
        \draw[black, thick, Latex-Latex] (D) -- (E);
        \draw[black, thick, -Latex] (D.315) arc (-135:135:10pt);

        \begin{scope}[shift={(6, 0)}]
            \node (X) at (0, -1)[draw=MidnightBlue, circle, very thick, MidnightBlue, fill=MidnightBlue!20] {\(X\)};

            \node (Y) at (2, -1)[draw=MidnightBlue, circle, very thick, MidnightBlue, fill=MidnightBlue!20] {\(Y\)};

            \node (Z) at (4, -1)[draw=MidnightBlue, circle, very thick, MidnightBlue, fill=MidnightBlue!20] {\(Z\)};

            \draw[black, thick, -Latex] (X) -- (Y);
            \draw[black, thick, Latex-Latex] (Y) -- (Z);
            \draw[black, thick, -Latex] (X.45) arc (-45:225:10pt);
            \draw[black, thick, -Latex] (Z.45) arc (-45:225:10pt);
        \end{scope}

        \draw[red!50, very thick, dashed] (A) to[out=90, in=90] (X);
        \draw[orange!50, very thick, dashed] (B) to[out=0, in=110] (Z);
        \draw[teal!50, very thick, dashed] (C) to[out=270, in=270] (X);
        \draw[blue!50, very thick, dashed] (D) to[out=270, in=225] (Z);
        \draw[violet!50, very thick, dashed] (E) to[out=0, in=135] (Y);
    \end{tikzpicture}
    \caption{A frame p-morphism between a frame \((W, R)\) with worlds \(W = \{A, B, C, D, E\}\) and a frame \((W', R')\) with worlds \(W' = \{X, Y, Z\}\).}
    \label{fig:Ch10-frame-p-morphism}
\end{figure}


Let \(v\) and \(v'\) be valuations mapping proposition letters to subsets of \(W\) and \(W'\) respectively. If \(f\) is a frame p-morphism from \((W, R)\) to \((W', R')\) and \(w \in v(p) \iff f(w) \in v'(p)\) for all worlds \(w\) and proposition letters \(p\), then \(f\) is said to be a \emph{model p-morphism} from model \((W, R, v)\) to \((W', R', v')\). This is illustrated in Figure \ref{fig:Ch10-model-p-morphism}.



\begin{figure}[H]
    \centering
    \begin{tikzpicture}[scale=1.5]
        \node (A) at (0, 0)[draw=MidnightBlue, circle, very thick, MidnightBlue, fill=MidnightBlue!20] {\(A_{\{\neg p, q\}}\)};

        \node (B) at (2, 0)[draw=MidnightBlue, circle, very thick, MidnightBlue, fill=MidnightBlue!20] {\(B_{\{p, q\}}\)};

        \node (C) at (0, -2)[draw=MidnightBlue, circle, very thick, MidnightBlue, fill=MidnightBlue!20] {\(C_{\{\neg p, q\}}\)};

        \node (D) at (2, -2)[draw=MidnightBlue, circle, very thick, MidnightBlue, fill=MidnightBlue!20] {\(D_{\{p, q\}}\)};

        \node (E) at (1, -1)[draw=MidnightBlue, circle, very thick, MidnightBlue, fill=MidnightBlue!20] {\(E_{\{p, \neg q\}}\)};


        \draw[black, thick, Latex-Latex] (A) -- (C);
        \draw[black, thick, -Latex] (A) -- (E);
        \draw[black, thick, Latex-Latex] (B) -- (E);
        \draw[black, thick, Latex-Latex] (B) -- (D);
        \draw[black, thick, -Latex] (C) -- (E);
        \draw[black, thick, Latex-Latex] (D) -- (E);
        \draw[black, thick, -Latex] (D.315) arc (-135:135:10pt);

        \begin{scope}[shift={(5, 0)}]
            \node (X) at (0, -1)[draw=MidnightBlue, circle, very thick, MidnightBlue, fill=MidnightBlue!20] {\(X_{\{\neg p, q\}}\)};

            \node (Y) at (2, -1)[draw=MidnightBlue, circle, very thick, MidnightBlue, fill=MidnightBlue!20] {\(Y_{\{p, \neg q\}}\)};

            \node (Z) at (4, -1)[draw=MidnightBlue, circle, very thick, MidnightBlue, fill=MidnightBlue!20] {\(Z_{\{p, q\}}\)};

            \draw[black, thick, -Latex] (X) -- (Y);
            \draw[black, thick, Latex-Latex] (Y) -- (Z);
            \draw[black, thick, -Latex] (X.45) arc (-45:225:10pt);
            \draw[black, thick, -Latex] (Z.45) arc (-45:225:10pt);
        \end{scope}

        \draw[red!50, very thick, dashed] (A) to[out=90, in=90] (X);
        \draw[orange!50, very thick, dashed] (B) to[out=0, in=110] (Z);
        \draw[teal!50, very thick, dashed] (C) to[out=270, in=270] (X);
        \draw[blue!50, very thick, dashed] (D) to[out=270, in=225] (Z);
        \draw[violet!50, very thick, dashed] (E) to[out=0, in=135] (Y);
    \end{tikzpicture}
    \caption{A model p-morphism between a model with worlds \(W = \{A, B, C, D, E\}\) and a model with worlds \(W' = \{X, Y, Z\}\). Valuations for each model are represented using subscripted curly brackets in each world.}
    \label{fig:Ch10-model-p-morphism}
\end{figure}

From this we can prove the following theorem on model p-morphisms.
%
\begin{theorem}
    Let \(f\) be a model p-morphism from model \((W, R, v)\) to model \((W', R', v')\). For any modal formula \(\phi\) and world \(w \in W\), we have
    %
    \[W, R, v, w \models \phi \iff W', R', v', f(w) \models \phi\]
\end{theorem}
\begin{proof}
    By structural induction on \(\phi\), with the syntax of modal formulas defined as follows.
    %
    \begin{align*}
        \text{proposition} &:= p \;|\; q \;|\; r \cdots\\
        \text{formula} &:= \text{proposition} \;|\; \neg\text{formula} \;|\; (\text{formula} \lor \text{formula}) \;|\; \Diamond \text{formula}
    \end{align*}

    \textbf{Base case.} For any proposition letter \(p\), we have
    %
    \begin{align*}
        W, R, v, w \models p &\iff w \in v(p)\\
        &\iff f(w) \in v'(p) \tag{by definition of model p-morphism}\\
        &\iff W', R', v', f(w) \models p
    \end{align*}
    %
    which completes the base case.

    \textbf{Step case for ``\(\neg\)formula''.} Assume \(W, R, v, w \models \phi \iff W', R', v', f(w) \models \phi\) for some formula \(\phi\). Then
    %
    \begin{align*}
        W, R, v, w \models \neg\phi &\iff W, R, v, w \not\models \phi\\
        &\iff W', R', v', f(w) \not\models \phi \tag{by induction hypothesis}\\
        &\iff W', R', v', f(w) \models \neg\phi
    \end{align*}
    
    \textbf{Step case for ``formula \(\lor\) formula''.} Assume
    %
    \begin{align*}
        W, R, v, w \models \phi &\iff W', R', v', f(w) \models \phi\\
        W, R, v, w \models \psi &\iff W', R', v', f(w) \models \psi
    \end{align*}
    %
    for some formulas \(\phi\) and \(\psi\). Then
    %
    \begin{align*}
        W, R, v, w \models \phi\lor\psi &\iff W, R, v, w \models \phi \text{ or } W, R, v, w \models \psi\\
        &\iff W', R', v', f(w) \models \phi \text{ or } W', R', v', f(w) \models \psi\\ \tag{by induction hypothesis}\\
        &\iff W', R', v', f(w) \models \phi\lor\psi
    \end{align*}
    
    \textbf{Step case for ``\(\Diamond\)formula''.} Assume \(W, R, v, w \models \phi \iff W', R', v', f(w) \models \phi\) for some formula \(\phi\). We want to prove that
    %
    \[W, R, v, w \models \Diamond\phi \iff W', R', v', f(w) \models \Diamond\phi\text{.}\]
    
    \((\Rightarrow).\)
    \begin{align*}
        W, R, v, w \models \Diamond\phi &\Rightarrow \text{there is some } w' \in W \text{ where } Rww' \text{ and } W, R, v, w' \models \phi\\
        &\Rightarrow \text{there is some } w' \in W \text{ where } (f(w), f(w')) \in R' \text{ and } W', R', v', f(w') \models \phi \tag{by induction hypothesis and definition of model p-morphism}\\
        &\Rightarrow W', R', v', f(w) \models \Diamond\phi
    \end{align*}

    \((\Leftarrow).\)
    \begin{align*}
        W', R', v', f(w) \models \Diamond\phi &\Rightarrow \text{there is some } x \in W' \text{ where } (f(w), x) \in R' \text{ and } W', R', v', x \models \phi\\
        &\Rightarrow \text{there is some } x_0 \in W \text{ where } (f(w), f(x_0)) \in R' \text{ and } W', R', v', f(x_0) \models \phi \tag{by definition of model p-morphism}\\
        &\Rightarrow \text{there is some } x_0 \in W \text{ where } (w, x_0) \in R \text{ and } W', R', v', f(x_0) \models \phi \tag{by definition of model p-morphism}\\
        &\Rightarrow \text{there is some } x_0 \in W \text{ where } (w, x_0) \in R \text{ and } W, R, v, x_0 \models \phi \tag{by induction hypothesis}\\
        &\Rightarrow W, R, v, w \models \Diamond\phi
    \end{align*}

    \textbf{Conclusion.} By principles of structural induction, the theorem is proved.
\end{proof}



\subsection{The class of irreflexive frames is not modally definable}

A frame \((W, R)\) is said to be \emph{irreflexive} if for all worlds \(w\) we have \((w, w) \in R\).

\begin{theorem}
    There is no modal formula that defines irreflexive flames. In other words, there is no formula \(\phi\) for which
    %
    \[(W, R) \models\phi \iff (W, R) \text{ is irreflexive.}\]
\end{theorem}
\begin{proof}
    By contradiction. Suppose there is a modal formula \(\phi\) that satisfies
    %
    \[(W, R) \models\phi \iff (W, R) \text{ is irreflexive.}\]
    %
    Consider the two-world irreflexive frame
    %
    \[\mathcal{F}_2 = (\{a, b\}, \{(a, b), (b, a)\})\]
    %
    and the one-world reflexive frame
    %
    \[\mathcal{F}_1 = (\{c\}, \{(c, c)\})\text{.}\]

    Let \(f : \mathcal{F}_2 : \mathcal{F}_1\) be a p-morphism where \(f(a) = f(b) = c\).

    Since \(\mathcal{F}_1\) is not irreflexive, we have \(\mathcal{F}_1 \not\models\phi\). Hence, there is some valuation \(v_1\) for which \(\mathcal{F}_1, c, v_1 \not\models\phi\). Define a valuation \(v_2\) over \(\mathcal{F}_2\) as
    %
    \[v_2 (p) = \begin{cases}
        \{a, b\} \text{ if } v(p) = \{c\}\\
        \emptyset \hspace{2em}\text{ if } v(p) = \emptyset
    \end{cases}\text{.}\]
    %
    Then \(f\) is a model p-morphism from \((\mathcal{F}_2, v_2)\) to \((\mathcal{F}_1, v_1)\). Hence \(\mathcal{F}_2, x, v_2 \not\models\phi\) for all \(x \in \{a, b\}\). This means that \(\mathcal{F}_2 \not\models\phi\), which is absurd as \(\mathcal{F}_2\) is irreflexive.
\end{proof}