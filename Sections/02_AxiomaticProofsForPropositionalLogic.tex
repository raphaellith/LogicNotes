\section{Axiomatic Proofs for Propositional Logic}

A \emph{proof system} is a system for determining the validity of formulas.

An obvious system would be to construct a truth table and check that all rows give a true result. However, this naive approach has an exponential time complexity\footnote{Using this system, checking the validity of a formula with \(n\) proposition symbols requires \(2^n\) computations.}, meaning that it will become increasingly impractical as more and more propositions are introduced.

To alleviate this issue, we shall introduce a different approach called a \emph{Hilbert-style proof system}. This is an \emph{axiomatic proof system} in which theorems are generated using axioms and inference rules.



\subsection{Hilbert-style proof system}

Firstly, we limit our propositional language to only use the connectives \(\neg\) and \(\rightarrow\). Double negations are prohibited.

Moreover, we will note some \emph{axioms} that are known to be valid, and then try to derive other valid formulas from the axioms. Below we list three examples of \emph{schemas}, from which axioms may be obtained by substituting any formulas in place of \(p\), \(q\) and \(r\).
%
\begin{enumerate}[I.]
    \item \(p \rightarrow (q \rightarrow p)\)
    \hfill (implication is true if consequent is true)
    \label{Ch02-axiom-I}
    
    \item \((p \rightarrow (q \rightarrow r)) \rightarrow ((p \rightarrow q) \rightarrow (p \rightarrow r))\)
    \hfill (implication chain as hypothetical syllogism)
    \label{Ch02-axiom-II}
    
    \item \((\neg p \rightarrow \neg q) \rightarrow (q \rightarrow p)\)
    \hfill (contrapositive)
    \label{Ch02-axiom-III}
\end{enumerate}

Axioms on their own are insufficient in establishing a proof system. We also need \emph{inference rules}, which stipulate how conclusions can be derived from premises. One of the main inference rules is \emph{modus ponens}, which states that if you have proved both the formula \(\phi\) and the implication \((\phi\rightarrow\psi)\), then you may deduce the conclusion \(\psi\).
%
\[
    \infer{\psi}{
        \phi
        &
        (\phi\rightarrow\psi)
    }
    %
    \tag{modus ponens}
\]

In this system, a \emph{proof} is a sequence of formulas
%
\[\phi_0,\; \phi_1,\; \phi_2,\; \cdots,\; \phi_n\]
%
such that for each \(i \leq n\), the formula \(\phi_i\) is either
%
\begin{itemize}
    \item an axiom; or
    \item obtained from two previous formulas \(\phi_j\) and \(\phi_k\) in the sequence via modus ponens (for some \(j, k < i\)).
\end{itemize}
%
If such a proof exists, then the final formula \(\phi_n\) is called a \emph{theorem} and we may write \(\vdash \phi_n\).

\begin{figure}[H]
    \centering
    \begin{tikzpicture}
        \foreach \i in {0,1,3} {
            \node (\i) at (2*\i, 0)[circle, draw=BrickRed, very thick, BrickRed, fill=BrickRed!20] {\(\phi_\i\)};

            \draw node[yshift=-0.75cm, BrickRed] at (2*\i, 0) {Axiom};
        }

        \foreach \i in {2,4,5} {
            \node (\i) at (2*\i, 0)[circle, draw=MidnightBlue, very thick, MidnightBlue, fill=MidnightBlue!20] {\(\phi_\i\)};
        }

        \draw[BurntOrange, very thick] (0) -- (1);
        \draw[BurntOrange, very thick, -Latex, rounded corners] (1, 0) -- (1, 1.5) -- (4,1.5) node[pos=0.5, above]{Modus ponens} -- (2);
        
        \draw[OliveGreen, very thick] (2) -- (3);
        \draw[OliveGreen, very thick, -Latex, rounded corners] (5, 0) -- (5, -1.5) -- (8,-1.5) node[pos=0.5, below]{Modus ponens} -- (4);
        
        \draw[Fuchsia, very thick] (3) -- (4);
        \draw[Fuchsia, very thick, -Latex, rounded corners] (7, 0) -- (7, 1.5) -- (10,1.5) node[pos=0.5, above]{Modus ponens} -- (5);

        \draw[black, ultra thick, -Latex] (-0.5, -3) -- (10, -3) node[right] {\textbf{Proof}};
    \end{tikzpicture}
    \caption{In a proof, every formula must be either an axiom, or derived from previous formulas via modus ponens.}
    \label{fig:Ch02-proof}
\end{figure}


For example, the theorem
%
\[\vdash (p \rightarrow p)\]
%
may be proved using the above proof system as follows.
%
\begin{enumerate}
    \item \((p \rightarrow ((p \rightarrow p) \rightarrow p)) \rightarrow ((p \rightarrow (p \rightarrow p)) \rightarrow (p \rightarrow p))\)
    \hfill (Axiom \ref{Ch02-axiom-I}, replacing \(p, q, r\) by \(p, (p \rightarrow p), p\))

    \item \(p \rightarrow ((p \rightarrow p) \rightarrow p)\)
    \hfill (Axiom \ref{Ch02-axiom-II}, replacing \(p, q\) by \(p, (p \rightarrow p)\))

    \item \((p \rightarrow (p \rightarrow p)) \rightarrow (p \rightarrow p)\)
    \hfill (modus ponens, via 1 and 2)

    \item \(p \rightarrow (p \rightarrow p)\)
    \hfill (Axiom \ref{Ch02-axiom-I}, replacing \(p, q\) by \(p, p\))

    \item \(p \rightarrow p\)
    \hfill (modus ponens, via 3 and 4)
\end{enumerate}



To include double negations and other connectives like \(\land\) and \(\lor\), we may add more axioms to our proof system.
%
\begin{enumerate}[I.]
    \setcounter{enumi}{3}
    \item \(p \rightarrow \neg\neg p\) and \(\neg\neg p \rightarrow p\)
    \hfill (double negation)
    \label{Ch02-axiom-IV}
    
    \item \((p \lor q) \rightarrow (\neg p \rightarrow q)\) and \((\neg p \rightarrow q) \rightarrow (p \lor q)\)
    \hfill (implication as disjunction)
    \label{Ch02-axiom-V}
    
    \item \((p \land q) \rightarrow \neg(p \rightarrow \neg q)\) and \(\neg(p \rightarrow \neg q) \rightarrow (p \land q)\)
    \hfill (implication as conjunction)
    \label{Ch02-axiom-VI}
\end{enumerate}



\subsection{Proofs with assumptions and the principle of explosion}

Let \(\Gamma\) be a set of \emph{assumptions}, i.e. formulas that are assumed to be true. Under these assumptions, a proof is defined as a sequence of formulas
%
\[\phi_0,\; \phi_1,\; \phi_2,\; \cdots \phi_n\]
%
such that for each \(i \leq n\), the formula \(\phi_i\) is either
%
\begin{itemize}
    \item an axiom;
    \item an assumption \(\phi_i \in \Gamma\); or
    \item obtained from two previous formulas \(\phi_j\) and \(\phi_k\) in the sequence via modus ponens (for some \(j, k < i\)).
\end{itemize}
%
If such a proof exists, then we may write \(\Gamma \vdash \phi_n\).

For example, given the set of assumptions \(\Gamma = \{p\}\), we may prove that \(q \rightarrow p\) using the Hilbert-style proof system, as demonstrated below.
%
\begin{enumerate}
    \item \(p \rightarrow (q \rightarrow p)\)
    \hfill (Axiom I)

    \item \(p\)
    \hfill (Assumption)

    \item \(q \rightarrow p\)
    \hfill (modus ponens, via 1 and 2)
\end{enumerate}


Proving with assumptions can be quite tricky due to the \emph{principle of explosion}\footnote{This principle is sometimes referred to in Latin as \textit{ex falso quodlibet}, which literally translates to ``from falsehood, anything [follows]''.}, which states that any statement can be proven from a contradiction. In other words, it is possible to prove any given statement, true or false, using a proof system as long as at least one of the assumptions in \(\Gamma\) is false.

We shall illustrate this principle as follows. Let \(\Gamma\) be the set containing the invalid assumption \(\neg(q \rightarrow q)\). We will use the Hilbert-style proof system to prove an arbitrary formula \(p\) under this assumption.
%
\begin{enumerate}
    \setcounter{enumi}{4}
    \item \(q \rightarrow q\)
    \hfill (proven previously)

    \item \((q \rightarrow q) \rightarrow \neg\neg(q \rightarrow q)\)
    \hfill (Axiom \ref{Ch02-axiom-IV}, replacing \(p\) by \(q\))

    \item \(\neg\neg(q \rightarrow q)\)
    \hfill (modus ponens, via 5 and 6)

    \item \(\neg\neg(q \rightarrow q) \rightarrow (\neg p \rightarrow \neg\neg(q \rightarrow q))\)
    \hfill (Axiom \ref{Ch02-axiom-I}, replacing \(p, q\) by \(\neg\neg(q \rightarrow q), \neg p\))

    \item \(\neg p \rightarrow \neg\neg(q \rightarrow q)\)
    \hfill (modus ponens, via 7 and 8)

    \item \((\neg p \rightarrow \neg\neg(q \rightarrow q)) \rightarrow (\neg (q \rightarrow q) \rightarrow p)\)
    \hfill (Axiom \ref{Ch02-axiom-III}, replacing \(p, q\) by \(p, \neg\neg(q \rightarrow q)\))

    \item \(\neg (q \rightarrow q) \rightarrow p\)
    \hfill (modus ponens, via 9 and 10)

    \item \(\neg (q \rightarrow q)\)
    \hfill (assumption)

    \item \(p\)
    \hfill (modus ponens, via 11 and 12)
\end{enumerate}



\subsection{Soundness, completeness and termination}

A proof system is said to be \emph{sound} if it can only prove valid theorems. In other words, anything proven using a sound system must be valid.
%
\begin{equation}
    \underbrace{\vdash \phi}_{\text{proven}} \implies \underbrace{\models \phi}_{\text{valid}} \tag{soundness}
\end{equation}

Conversely, a proof system is said to be \emph{complete} if it can prove any given valid theorem. In other words, if a formula is valid, it must be possible to prove it under a complete system.
%
\begin{equation}
    \underbrace{\models \phi}_{\text{valid}} \implies \underbrace{\vdash \phi}_{\text{proven}} \tag{completeness}
\end{equation}

The main problem with the Hilbert-style proof system is that although it is relartively easy to check that a proof of a formula is correct, there is no systematic way for efficiently constructing proofs.

Moreover, even if a system is sound and complete, we don't know how long the proof for a given formula might be. Since it is impossible for us to check all the possibilities to see if a proof exists, testing the validity of a formula remains undecidable --- there is no effective method for determining validity that terminates in finite time.