\section{Compactness}

\subsection{Compactness theorem}

Proofs are finite. If for some formula \(\phi\) and some set of assumptions \(\Sigma\) we have \(\Sigma\vdash\phi\), then we must also have \(\Sigma_0 \vdash\phi\) for some finite subset \(\Sigma_0\) of \(\Sigma\).

Moreover, we say that \(\Sigma\) is \emph{inconsistent} if \(\Sigma\vdash(p\land\neg p)\). If there is no such proof, we say that \(\Sigma\) is \emph{consistent}.

\begin{theorem}
    A set of formulas \(\Sigma\) has a model if and only if each finite subset of \(\Sigma\) has a model.
\end{theorem}
\begin{proof}
    (\(\Rightarrow\)): Assume that \(\Sigma\) has a model \((D, I)\). Then trivially every finite subset of \(\Sigma\) must also have the same model \((D, I)\).

    (\(\Leftarrow\)): We prove the contrapositive --- if \(\Sigma\) does not have a model, then there is some finite subset of \(\Sigma\) that does not have a model either.

    If \(\Sigma\) has no model, then \(\Sigma\models\bot\). By strong completeness, this set must be inconsistent, with \(\Sigma\vdash\bot\). Since proofs are finite, there exists some finite subset \(\Sigma_0 \subseteq \Sigma\) such that \(\Sigma_0 \vdash\bot\). By soundness, it follows that \(\Sigma_0 \models\bot\), i.e. \(\Sigma\) does not have a model. Hence proved.
\end{proof}



\subsection{Applying the compactness theorem}

\subsubsection{Connectedness}

Let \(E\) be a binary relation denoting the edges of a graph. Let \(=\) denote equality.

For \(k\geq 0\), we say that there is a \emph{path} of length \(k\) from \(x\) to \(y\) if there is a sequence
%
\[x_0, x_1, \cdots, x_k\]
%
where \(x = x_0\), \(y = x_k\) and \((x_i, x_i+1) \in E\) for all \(i < k\).

We can then write the following formulas.
%
\begin{itemize}
    \item There is a path of length \(0\) from node \(x\) to node \(y\).
    \[x = y\]

    \item There is a path of length \(1\) from node \(x\) to node \(y\).
    \[E(x, y)\]

    \item There is a path of length \(2\) from node \(x\) to node \(y\).
    \[\exists z\; (E(x, z) \land E(z, y))\]

    \item There is a path of length \(3\) from node \(x\) to node \(y\).
    \[\exists w\; ((\exists z\; (E(x, z) \land E(z, w))) \land E(w, y))\]
\end{itemize}
%
Let \(P_n (x, y)\) be the statement ``There is a path of length \(n\) from node \(x\) to node \(y\)''. In general, \(P_n (x, y)\) can be expressed in first-order logic as follows.
%
\begin{align*}
    P_0 (x, y) &= ``x = y"\\
    P_{n+1} (x, y) &= ``\exists m\; (P_n (x, m) \land E(m, y))"
\end{align*}

A directed graph \((G, E)\) consists of a set of nodes \(G\) and a binary relation \(E \subseteq G \times G\).  Such a graph \((G, E)\) is said to be \emph{connected} if for all \(x, y \in G\) there is a path of finite length \(k\) from \(x\) to \(y\). How can we define this concept of connectedness using first-order logic?

A possible approach might be something like
%
\[
    \smashoperator{
        \mathop{
            \mathlarger{\mathlarger{\mathlarger{
                \lor
            }}}
        }
    }_{n \in \mathbb{N}}
    (
        P_n(x, y)
    )
\]
%
or
%
\[
    \exists n\; P_n(x, y)\text{.}
\]
However, neither of these formulas are acceptable. The first formula is problematic because first-order logic does not allow an infinite number of connectives; the second formula cannot be accepted because \(\exists n\) assumes the domain of \(\mathbb{N}\), when the domain is actually \(G\).

In fact, it is impossible to define connectedness using first-order logic, as we will prove below.

\begin{theorem}[First-order logic cannot define connectedness]
    There is no first-order formula \(\phi\) such that a graph \(G\) satisfies \(\phi\) if and only if \(G\) is connected.
\end{theorem}
\begin{proof}
    By contradiction. Let \(L = (C, F, P)\) be the first-order lanaguage where \(C = \{c, d\}\), \(F = \emptyset\) and \(P = \{E, =\}\), with \(E\) representing the edge relation. Suppose \(\Sigma\) is a theory such that for any graph \(G\), we have \(\Sigma\vdash G\) if and only if \(G\) is a connected graph.

    Let \(P_n (x, y)\) be a first-order formula expressing the statement ``there is a path of length \(n\) from node \(x\) to node \(y\)'', as defined earlier.

    Consider the theory
    %
    \[\Sigma' =
    \underbrace{\Sigma}_{
        \substack{
            G \text{ is}\\
            \text{connected}
        }
    }
    \cup
    \underbrace{\{\neg\phi_n (c, d) : n\in\mathbb{N}\}}_{
        \substack{
            \text{no path of any length} \\
            \text{from } c \text{ to } d\\
            \text{(\(G\) is disconnected)}
        }
    }\]
    %
    which does not have a model, as a graph cannot be both connected and disconnected. This can be denoted as \(\Sigma' \models\bot\).

    Now consider a finite subset \(\Sigma'_0\) of \(\Sigma'\). This gives us
    %
    \[\Sigma'_0 = \Sigma_0 \cup \{\neg \phi_n (c, d) : n\in S\}\]
    %
    where \(\Sigma_0\) is a finite subset of \(\Sigma\) and \(S\) is a finite set of natural numbers.

    We show that \(\Sigma'_0\) has a model. Since \(S\) is finite, we can find some \(N \in\mathbb{N}\) that is larger than any element in \(S\). Construct a graph \(G\) where
    %
    \begin{itemize}
        \item the nodes are \(\{0, 1, 2, \cdots, N+1\}\);
        \item edges are present between consecutive numbers only; and
        \item \(c = 0\) and \(d = N+1\).
    \end{itemize}
    %
    Hence the shortest path between \(c\) and \(d\) has length \((N+1)\). This is a model of \(\Sigma'_0\).

    By the compactness theorem, since any finite subset of \(\Sigma'\) has a model, \(\Sigma'\) itself must also have a model. This contradicts our earlier proposition that \(\Sigma'\) does not have a model.
\end{proof}